\section{Анализ предметной области}
\subsection{Описание предметной области}

Агентство недвижимости (АН) — это организация, предоставляющая посреднические услуги при совершении сделок с недвижимостью. Деятельность АН охватывает широкий спектр операций, включая:
\begin{itemize}
	\item покупка и продажа недвижимости: поиск объектов, соответствующих требованиям клиентов, организация просмотров, переговоры о цене, оформление договоров купли-продажи;
	\item аренда (долгосрочная и краткосрочная): подбор объектов для арендаторов, поиск арендаторов для собственников, составление договоров аренды;
	\item управление недвижимостью (Property Management): обслуживание объектов недвижимости, контроль платежей, ремонт и техническое обслуживание;
	\item консультационные услуги: оценка стоимости недвижимости, юридическое сопровождение сделок, помощь в получении ипотеки и т.д.
\end{itemize}

Эффективная работа АН предполагает обработку и анализ большого объёма информации, относящейся к различным категориям:
\begin{itemize}
\item объекты недвижимости: квартиры, дома, земельные участки, коммерческая недвижимость (офисы, магазины, склады) и т.д. Информация об объектах включает адрес, характеристики (площадь, количество комнат, материалы стен, год постройки и т.д.), фотографии, цену, описание, юридические документы;
\item клиенты: потенциальные покупатели, продавцы, арендаторы и арендодатели. Информация о клиентах включает контактные данные, предпочтения, требования к недвижимости, историю сделок;
\item сотрудники: риелторы, менеджеры, юристы, оценщики и другие специалисты. Информация о сотрудниках включает контактные данные, специализацию, историю работы, комиссионные;
\item сделки: договоры купли-продажи, аренды, оказания услуг. Информация о сделках включает сведения об объекте, клиентах, сотрудниках, дате заключения, цене, условиях оплаты, комиссии АН;
\item рекламные кампании и источники: информация о размещенных объявлениях, каналах привлечения клиентов, результатах рекламных кампаний.
\end{itemize}

\subsection{Цели и задачи агентства недвижимости}

Основными целями АН являются:
\begin{itemize}
\item максимизация прибыли: за счет успешных сделок и оказания качественных услуг;

\item удовлетворение потребностей клиентов: обеспечение оптимального подбора недвижимости и сопровождение сделок;

\item повышение эффективности работы сотрудников: оптимизация рабочих процессов, сокращение временных затрат и повышение производительности;

\item	расширение клиентской базы: привлечение новых клиентов и удержание существующих;

\item	укрепление репутации: Предоставление надежных и профессиональных услуг.
\end{itemize}
Для достижения этих целей АН выполняет следующие задачи:
\begin{itemize}
\item	поиск и привлечение клиентов: Использование различных каналов рекламы и маркетинга;

\item поиск и оценка объектов недвижимости: анализ рынка, поиск подходящих объектов, оценка их стоимости;

\item организация просмотров и переговоров: встречи с клиентами, организация показов объектов, ведение переговоров о цене и условиях сделки;

\item юридическое сопровождение сделок: проверка юридической чистоты объектов, подготовка и оформление договоров, регистрация сделок;

\item	финансовый контроль: контроль платежей, ведение бухгалтерского учета, расчет комиссионных;

\item	анализ рынка: сбор и анализ данных о рынке недвижимости, прогнозирование тенденций.
\end{itemize}

\subsection{Особенности предметной области, влияющие на проектирование БД}

При проектировании базы данных для АН необходимо учитывать следующие особенности:
\begin{itemize}
\item	большой объем данных: база данных должна быть способна обрабатывать большие объемы данных о недвижимости, клиентах и сделках;

\item	необходимость поиска и фильтрации данных: пользователям нужна возможность быстрого поиска объектов по различным критериям (цена, площадь, местоположение, количество комнат и т. д.), а также фильтрации данных по различным параметрам;

\item	необходимость формирования отчетов: база данных должна обеспечивать возможность формирования различных отчетов, необходимых для анализа деятельности АН (отчеты о сделках, комиссионных, продажах и т. д.);

\item	безопасность данных: необходимо обеспечить защиту конфиденциальной информации о клиентах и сотрудниках;

\item	масштабируемость: база данных должна быть масштабируемой, чтобы учитывать рост агентства и увеличение объема обрабатываемой информации;

\item	интеграция с другими системами (возможно): интеграция с сайтом АН, CRM-системами, системами учета и отчетности.
\end{itemize}

\subsection{Анализ бизнес-процессов}
	
На основе анализа неформального описания предметной области были сформулированы бизнес-правила:
\begin{itemize}
\item	у каждого объекта недвижимости должен быть владелец;

\item	у каждого владельца должен быть телефон для связи;

\item	каждая сделка имеет определенный тип;

\item	при регистрации каждой сделки данные о продавце и покупателе обязательны;

\item	каждый объект недвижимости, сотрудник, владелец, покупатель, сделка должны иметь уникальный код ( ID);

\item	один владелец может иметь несколько квартир в собственности;

\item	необходимо корректно вводить данные во всех полях.
\end{itemize}
Ограничения целостности для таблицы ОБЪЕКТ НЕДВИЖИМОСТИ
\begin{itemize}
\item	код объекта недвижимости является уникальным для каждого объекта недвижимости, разрешены только цифры;

\item	количество комнат, цена- данные строки могут содержать только значения в виде цифр;

\item	недопустимы пустые значения во всех полях, кроме срока аренды.
\end{itemize}
Ограничения целостности для таблицы ПОКУПАТЕЛЬ/АРЕНДАТОР
\begin{itemize}
\item	код покупателя/арендатора является уникальным для каждого покупателя/арендатора, разрешены только цифры;

\item	паспортные данные, контактные данные- данные строки могут содержать только значения в виде цифр;

\item	фамилия, имя, отчество – строка символов, длиной до 50 символов. Может содержать только буквы русского алфавита;

\item	недопустимы пустые значения во всех полях, кроме отчества клиента.
\end{itemize}
Ограничения целостности для таблицы ПРОДАВЕЦ/АРЕНДОДАТЕЛЬ

Правила для контроля уникальности в ключевом поле и требования к типам данных и ограничения на допустимые значения данных во всех полях разрабатываются по аналогии с приведенными для таблицы ПОКУПАТЕЛЬ/АРЕНДОДАТЕЛЬ.

Ограничения целостности для таблицы ДОГОВОР АРЕНДЫ
\begin{itemize}
\item	код договора аренды является уникальным для каждого договора, разрешены только цифры;

\item	дата заключения договора- календарная дата;

\item	при заключении договора обязательно должны быть заполнены данные о арендодателе, арендаторе, сотруднике агентства, объекте недвижимости;

\item	недопустимы пустые значения во всех полях.
\end{itemize}
Ограничения целостности для таблицы ДОГОВОР ПРОДАЖИ

	Правила для контроля уникальности в ключевом поле и требования к типам данных и ограничения на допустимые значения данных во всех полях разрабатываются по аналогии с приведенными для таблицы ДОГОВОР АРЕНДЫ.


